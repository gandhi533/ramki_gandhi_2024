\documentclass[a4paper,11pt]{article}
\begin{document}

\begin{center}
ALL CODES ARE IN FORTRAN 77
\end{center}
\begin{center}
  PROBLEM I
\end{center}
\begin{enumerate}
\item The codes were written with some non-dimensional variables different
  from those in paper. They are given below. Primes are used for these. Nomenclature for
  dimensional variables is same as in the paper.  The different variables are:
  \[\tau^\prime= \frac{D
    t}{L^2},\Lambda^\prime=\frac{|I|L}{DC_{ref}nF}\]
  Hence, these are related to those in the paper as follows:
  \[\tau=\tau^\prime\frac{D_1}{D}, \Lambda=\Lambda^\prime\frac{D}{D_1}\]
\item  
  Some are defined in the code only. 
  \[{\rm kay\_1}=\kappa_1^\prime=\kappa_1\frac{D_1}{D},\,\, {\rm kay\_2}=\kappa_2^\prime=\kappa_3\frac{D_1}{D}\]
  \[{\rm cap\_gamma}=\frac{|I|L}{DnF}, {\rm c\_not}=C_{ref},\,\,\,{\rm
    hence}\,\,\,\frac{\rm cap\_gamma}{\rm c\_not} = \Lambda^\prime\]
  delta\_tau = delta\_$\tau^\prime$ is the increment in time step.
\item Other variables in the code are same as in the paper:
  deltai=$\Delta_i$,alambda(k)=$\lambda_k$, ai(k),bi(k) are the
  coefficients in the $k^{th}$ eigenfunction. 
\end{enumerate}
\vspace{.1in}

\noindent {\bf Code named iteration\_uniform\_ic.f} 

This is the code to compute $\kappa_1^\prime$ and $\kappa_3^\prime$
as a function of $\tau^\prime$. 

The input data are in these files:
\begin{enumerate}
  \item input\_plotcheq.dat contains geometry of the cell
  \item  output\_normalized\_vector.dat contains information
    on eigenvalues and normalized  eigenfunction coefficients
\item inner\_prod\_uniform\_ic\_diff\_5bc.dat contains data on the
  inner product of the uniform initial condition with the
  eigenfunctions.
\item operating.dat contains information for running program and
    also to change $\Lambda^\prime$ through cap\_gamma.
\end{enumerate}
{\it The first three data files are common between problem I and
  III. They are placed in the directory code\_problem\_I to avoid
  unnecessary repetition.}
The output file is output\_iteration\_general.dat. Second column is
$\tau^\prime$. The last two columns are kay\_1 and kay\_2. The middle
columns are the second terms of the r.h.s. in the second
equation of eq.33 of the paper for each eigenfunction. This data is used 
The first term of the first equation of eq.33 of the paper is a linear
decreasing function of time as seen from eq.33. Both these are used to
construct concentration profile using the following code. 

\vspace{.1in}
\noindent {\bf Code named conc\_profile.f}

This constructs concentration profile using eq.32 of the paper. The
notation is same as described above and is a straight forward
summation. output\_conc\_profile.dat is the output file. The output is
$u$ vs. $\xi$ as different values of $\tau^\prime$.
\vspace{0.1in}

{\it All code files and data files are in the directory code\_problem\_I.}
\begin{center}
  PROBLEM III
\end{center}
Many items and files are common to problem I. Hence, only differences
will be mentioned here.
\begin{enumerate}
\item The new quantities defined make computation easier. These are
  udotz0(i) $=\left<{\bf u.Z}_o\right>(\tau^\prime)$ and udotz(i,j) $=\left<{\bf u.Z}_j\right>(\tau^\prime) $
\item cs1max and cs3max are $C_{s1,max}$ and $C_{s3,max}$ of paper.
  \item u4 and u5 are $u_4$ and $u_5$ of paper.
\end{enumerate}
\vspace{.1in}
\noindent {\bf Code named iteration\_uniform\_ic\_non\_linear.f}

This is the code to compute $\kappa_1^\prime$, $\kappa_3^\prime$, and $u_i(\xi)$
as a function of $\tau^\prime$.  conc\_profile.f was integrated into
iteration program in this code.

Only new input file is operating\_non\_linear.dat instead of
operating.dat as it needs data on lithium concentration in the active
materials. 

The output files are
\begin{enumerate}
  \item first three columns of output\_iteration\_general.dat are $i$
  (the time index),  kay\_1 and kay\_2.
  \item output\_li\_content.dat gives i (the time index) against
    $u_4$ and $u_5$  as a function of j
    (index of $\xi$).   
\item output\_conc\_profile.dat gives i (the time index) versus
  $u_1$,$u_2$,$u_3$ as a function of  j
  (index of $\xi$).
\end{enumerate}

{\it All code files and data files are in the directory code\_problem\_III.}

\vspace{0.1in}

\begin{center}
  PROBLEM II
\end{center}


\begin{enumerate}
\item The codes were written with some non-dimensional variables different
  from those in paper. They are given below. Primes are used for these. Nomenclature for
  dimensional variables is same as in the paper.  These are as follows:
  \[\tau^\prime= \frac{D_1
    t}{L_1^2},\Lambda^\prime=\frac{|I|L_1}{D_1C_{ref}nF}\]
  Hence, these are related to those in the paper as follows:
  \[\tau=\tau^\prime\frac{L^2_1}{L^2}, \Lambda=\Lambda^\prime\frac{L}{L_1}\]
\item   Some are defined in the code only.
 \[{\rm alpha}=\frac{L_1}{L-L_1}, {\rm beta(j)}=\lambda(j), {\rm cap\_lambda}=\Lambda^\prime, {\rm akappa}=\kappa_1^\prime,{\rm u1}=u_m, {\rm conc(i)}=u_1(\xi) \]
\[\kappa_1^\prime=\kappa_1\frac{L_1^2}{L^2}\]
 \[{\rm g(i)}=g(\tau) {\rm of\,\, eq.A.18}, {\rm
  u\_dot\_z\_zero(i)}=\left<{\bf u.Z}_o\right>(\tau),{\rm
  u\_dot\_zjxi(i)}=\sum_j\left<{\bf u.Z}_j\right>(\tau)Z_j(\xi)\]
\item const\_norm(j) are the normalization constants for
eigenfunctions, eq.(A.12) of paper.
\end{enumerate}
\vspace{.1in}

\noindent {\bf Code named program\_iteration\_v\_4.f} 

This is the iteration program. It generates results for
tau,i (index for time), u1(i), akappa(i), u\_dot\_z\_zero(i)  in the output file named
  output\_plot\_tau\_u1.dat. The input files are
  \begin{enumerate}
    \item input\_test\_ev.dat for eigen values
\item input\_iteration.dat for giving inputs including cap\_lambda for
  changing operating conditions.
\end{enumerate}
  \noindent {\bf Code named program\_conc\_xi\_v\_2.f}

  Reads output from previous program and generates concentration
  profiles using eq.41 of paper. The output is in the file 
output\_all\_xi\_tau.dat. It gives $u_1(\tau)$ at each value of $\xi$.\\


{\it All  code files  and data files are in the directory code\_program\_II.}
\end{document}
